% !TeX spellcheck = de_DE
\chapter{Abbildungen, Tabellen \& Vermischtes}

\section{Abbildungen}
Beispiel für eine Abbildung:
\begin{verbatim}
	\begin{figure}
  	  \centering
  	  \begin{subfigure}[h!]{0.15\linewidth}
    	\includegraphics[width=\linewidth]{example-image-a}
    	\caption{}
  	  \end{subfigure}
  	  \begin{subfigure}[h!]{0.15\linewidth}
    	\includegraphics[width=\linewidth]{example-image-b}
    	\caption{}
  	  \end{subfigure}\\
  	  \begin{subfigure}[h!]{0.3\linewidth}
    	\includegraphics[width=\linewidth]{example-image-c}
    	\caption{}
  	  \end{subfigure}
  	  \caption[Kurzbeschreibung für Abbildungsverzeichnis]{Lange Beschreibung,
  	  			die unter dem Bild angezeigt wird.}
  	  \label{Abb:beispiel}
	\end{figure}
\end{verbatim}

\begin{figure}
  \centering
  \begin{subfigure}[h!]{0.25\linewidth}
    \includegraphics[width=\linewidth]{example-image-a}
    \caption{}
  \end{subfigure}
  \begin{subfigure}[h!]{0.25\linewidth}
    \includegraphics[width=\linewidth]{example-image-b}
    \caption{}
  \end{subfigure}\\
  \begin{subfigure}[h!]{0.5\linewidth}
    \includegraphics[width=\linewidth]{example-image-c}
    \caption{}
  \end{subfigure}
  \caption[Kurzbeschreibung für Abbildungsverzeichnis]{Lange Beschreibung, die unter dem Bild angezeigt wird.}
  \label{Abb:beispiel}
\end{figure}

Zeilenumbrüchen funktionieren auch hier mit \textbackslash \textbackslash . Der leere Befehl \textbackslash caption\{\} wird benötigt um die Nummerierung \glqq{}(a)\grqq{} usw. einzufügen. Anstatt von \glqq{}example-image-a\grqq{} usw. muss der Pfad zum gewünschten Bild eingefügt werden.

\pagebreak % damit Abbildung und Tabelle hier nicht durcheinander geraten

\section{Tabellen}
Beispiel für eine Tabelle:
\begin{verbatim}
	\begin{table}
	\centering
	\caption[Kurzbeschreibung für Tabellenverzeichnis]{Lange Beschreibung,
				die über der Tabelle angezeigt wird.}
	\begin{tabular}{ccc}
	Spalte 1 & Spalte 2 & Spalte 3\\
	\hline 
	A & B & C \\
	D & E & F \\
	G & H & I \\
	\hline 
	\end{tabular}
	\label{Tab:beispiel}
	\end{table}
\end{verbatim}

\begin{table}
\centering
\caption[Kurzbeschreibung für Tabellenverzeichnis]{Lange Beschreibung, die über der Tabelle angezeigt wird.}
\begin{tabular}{ccc}
Spalte 1 & Spalte 2 & Spalte 3\\
\hline 
A & B & C \\
D & E & F \\
G & H & I \\
\hline 
\end{tabular}
\label{Tab:beispiel}
\end{table}

\section{Größen und Einheiten im Text}
Größen mit Einheiten sollten im Text so verwendet werden:
\begin{verbatim}
	\gls{kb} $= 1,38064852 10^{-23}$~m$^2$kgs$^{-2}$K$^{-1}$
	$l = 1.85$~\AA{}
\end{verbatim}
\gls{kb} $= 1,38064852 10^{-23}$~m$^2$kgs$^{-2}$K$^{-1}$\\
$l = 1.85$~\AA{}