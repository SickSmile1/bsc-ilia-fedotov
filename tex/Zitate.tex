% !TeX spellcheck = de_DE
\chapter{Zitate}

\section{Syntax}
Die Informationen über die zitierten Quellen müssen als \textasteriskcentered .bib-Dateien bereitgestellt und in \glqq{} main.tex\grqq{} geladen werden:
\begin{verbatim}
	\addbibresource{referenzen/Software.bib}
	\addbibresource{referenzen/Potentiale.bib}
\end{verbatim}
Dabei dürfen alle Quellen in einer Datei gespeichert sein, aber auch auf mehrere Dateien verteil werden.

Quellen können entweder einzeln zitiert werden, oder als Gruppe:
\begin{verbatim}
	\cite{Schlüssel}
	\cite{Schlüssel_1,Schlüssel_2,Schlüssel_3,Schlüssel_4}
\end{verbatim}

Um Zitate einzufügen wie im Beispiel unten sollte ein geschütztes Leerzeichen (\textasciitilde) verwendet werden, um Zeilenumbrüche am Zitat zu verhindern:
\begin{verbatim}
	LAMMPS~\cite{Plimpton1995}
\end{verbatim}

In das Dokument eingefügt wird die Bibliographie in \glqq{} main.tex\grqq{} mit
\begin{verbatim}
	\printbibliography
\end{verbatim}

\section{Anpassungsmöglichkeiten}
In \glqq{} main.tex\grqq{} gibt es verschiedene Möglichkeiten die Art der Zitierung zu beeinflussen.
\begin{verbatim}
	\usepackage[backend=biber,style=numeric,citestyle=numeric-comp,sorting=none]{biblatex}
\end{verbatim}
Hier wurde festgelegt die Zitate mit Zahlen zuzuordnen (\glqq{}style=numeric\grqq{}), mehrere Zitate an einer Stelle kompakt zu beschreiben (\glqq{}citestyle=numeric-comp\grqq{}) und die Quellen in der Reihenfolge aufzulisten in der sie im Text auftauchen anstatt sie zu sortieren (\glqq{}sorting=none\grqq{}).

\section{Beispiel - Softwares und Potentiale}
Als Beispiel sind hier einige Softwares genannt, die in unserer Gruppe häufig verwendet werde: LAMMPS~\cite{Plimpton1995}, OVITO~\cite{Stukowski2010}, ASE~\cite{HjorthLarsen2017}, matscipy~\cite{matscipy} und voro++~\cite{Rycroft2009}.
Häufig verwendete interatomare Potentiale für amorphen Kohlenstoff~\cite{Deringer2016,Pastewka:2013p205410,Liyanage2014} und CuZr metallisches Glas~\cite{Mendelev2009,Cheng2009}.