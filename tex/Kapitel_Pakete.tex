% !TeX spellcheck = de_DE
\chapter{Pakete}\label{Kap:Pakete}
Dieses Kapitel beschreibt einige der verwendeten Pakete. Vollständige Dokumentationen sind verlinkt.

\section{glossaries}\label{Abs:glossaries}
Dieses Paket ermöglicht es gleichzeitig eine Liste von Abkürzungen und eine Liste von Symbolen zu führen. In \glqq{}main.tex\grqq{} wird das Paket geladen:
\begin{verbatim}
	\usepackage[acronym,nonumberlist]{glossaries}
	\makeglossaries
\end{verbatim}

die Dateien mit den Listen eingebunden
\begin{verbatim}
	% einfach: \newacronym{Schl\"ussel}{Kurzform}{Langform}
% ausf\"uhrlich: \newacronym[plural=Kurzform (plural),firstplural=Langform (plural)]{Schl\"ussel}{Kurzform}{Langform}

\newacronym{ALU}{ALU}{Albert-Ludwigs-Universität}

\newacronym{IMTEK}{IMTEK}{Institut für Mikrosystemtechnik}

\newacronym{PDF}{PDF}{Portable Document Format}
	\newglossaryentry{kb}{name=\ensuremath{k_B},description=Boltzmann Konstante}
\newglossaryentry{bulkmod}{name=\ensuremath{K},description=Bulkmodul}

\newglossaryentry{density}{name=\ensuremath{\rho},description=Dichte}

\newglossaryentry{Tg}{name=\ensuremath{T_g},description=Glasübergangstemperatur}

\newglossaryentry{heaviside}{name=\ensuremath{\Theta},description=Heaviside Stufenfunktion}

\newglossaryentry{kronecker}{name=\ensuremath{\delta},description=Kroneckerdelta}

\newglossaryentry{poissonratio}{name=\ensuremath{\nu},description=Poissonzahl}
\newglossaryentry{pressure}{name=\ensuremath{p},description=Druck}

\newglossaryentry{temperature}{name=\ensuremath{T},description=Temperatur}

\newglossaryentry{youngsmod}{name=\ensuremath{E},description=Young's Modul}
\end{verbatim}

Um alle Abkürzungen in der Liste der Abkürzungen zu drucken, auch solche die nicht verwendet wurden, dient dieser Befehl:
\begin{verbatim}
	\glsaddallunused
\end{verbatim}
Bei der Erstellung des Dokuments kann dies hilfreich sein, um alle bereits definierten Abkürzungen im \gls{PDF} zu sehen.

Diese Befehlen fügen die Listen in das Dokument ein:
\begin{verbatim}
	\printglossary[type=\acronymtype,title=Abkürzungen]
	\printglossary[title=Symbole]
\end{verbatim}

Die Liste der Symbole wird in \glqq{}Symbole.tex\grqq{} definiert. Die Syntax ist
\begin{verbatim}
	\newglossaryentry{Schlüssel}{name=\ensuremath{Symbol},description=Beschreibungstext}
\end{verbatim}
(Der Befehl ensuremath erleichtert später die Verwendung des Symbols im Text und in Gleichungen. Würde hier eine \$-Umgebung verwendet, würden in Gleichungen Fehler auftreten.)

Abkürzungen werden in \glqq{}Abkuerzungen.tex\grqq{} definiert mit
\begin{verbatim}
	\newacronym{Schlüssel}{Kurzform}{Langform}
	\newacronym[plural=Kurzform (plural),firstplural=Langform (plural)]{Schlüssel}{Kurzform}{Langform}
\end{verbatim}
Die erste Variante definiert den Plural nicht ausdrücklich, d.~h. er wird automatisch erzeugt. Entspricht dies nicht der gewünschten Form, kann diese manuell definiert werden.

Abkürzungen und Symbole werden im Text mit dem Befehl
\begin{verbatim}
	\gls{Schlüssel}
\end{verbatim}
eingefügt. Dabei wird in den \{\} der Schlüssel für die jeweilige Abkürzung verwendet. Der Plural der Abkürzung wird mit
\begin{verbatim}
	\glspl{Schlüssel}
\end{verbatim}
eingefügt. Beispiele für ein Symbol im Text und in einer Gleichung, Singular und Plural Abkürzungen (\glqq{}PDF\grqq{} wurde bereits verwendet und erscheint deshalb hier in der Kurzform):
\begin{verbatim}
	\gls{density}
	\begin{equation}
		\gls{density} = \frac{m}{V}
		\label{Gl:dichte}
	\end{equation}
	\gls{IMTEK}
	\glspl{PDF}
\end{verbatim}
\gls{density}
\begin{equation}
	\gls{density} = \frac{m}{V}
	\label{Gl:dichte}
\end{equation}
\gls{IMTEK}\\
\glspl{PDF}

Die Langform einer Abkürzung wird nur bei der ersten Verwendung der Abkürzung gedruckt, danach die Kurzform. Um später erneut die Langform zu drucken (z.~B. in jedem Kapitel), kann der Zähler für einzelne oder alle Abkürzung zurück gesetzt werden:
\begin{verbatim}
	\glsreset{Schlüssel}
	\glsresetall
\end{verbatim}

Vor dem Erzeugen des \glspl{PDF} muss in einer Kommandozeile, im Verzeichnis mit \glqq{}main.tex\glqq{}, folgender Befehl ausgeführt werden (sonst fehlen die Listen im \gls{PDF}):
\begin{verbatim}
	makeglossaries main
\end{verbatim}
(\glqq{}main\grqq{} ist hier der Name der Hauptdatei, ohne die Endung \glqq{}tex\grqq{})

\hyperlink{https://ctan.org/pkg/glossaries?lang=de}{glossaries Dokumentation}

\section{cleveref}\label{Abs:cleveref}
Objekte innerhalb der Arbeit (Abbildungen, Kapitel, Gleichungen) können referenziert werden, wenn ihnen zuvor ein Schlüssel zugewiesen wurde. Dies erfolgt direkt am Objekt:
\begin{verbatim}
	\section{cleveref}\label{Abs:cleveref}
\end{verbatim}

Im Text kann dann auf dieses Objekt verwiesen werden:
\begin{verbatim}
	\cref{Abs:cleveref}
	\cref{Tab:beispiel}
	\cref{Abb:beispiel}
\end{verbatim}
\cref{Abs:cleveref}\\
\cref{Tab:beispiel}\\
\cref{Abb:beispiel}

Auch ein ganzer Bereich von Objekten kann referenziert werden:
\begin{verbatim}
	\crefrange{Abs:glossaries}{Abs:cleveref}
\end{verbatim}
\crefrange{Abs:glossaries}{Abs:cleveref}

(Es ist gute Praxis die Schlüssel so zu vergeben, dass daraus ersichtlich ist welcher Art das Objekt ist (Abbildung, Kapitel, etc.), da sich dadurch die Lesbarkeit des Quelltextes erhöht.)

\hyperlink{https://ctan.org/pkg/cleveref?lang=de}{cleveref Dokumentation}

\section{xcolor}
Dieses Paket erlaubt es die \colorbox{blue}{F}\colorbox{green}{a}\colorbox{yellow}{r}\colorbox{orange}{b}\colorbox{red}{e} von Schrift, Hintergrund, Rahmen oder ganzen Seiten zu ändern. Kommentare im Text können z.~B. farbig hinterlegt werden, um die Sichtbarkeit zu erhöhen:
\begin{verbatim}
	\colorbox{yellow}{hier Zitat einfügen}
\end{verbatim}
\colorbox{yellow}{hier Zitat einfügen}

\hyperlink{https://ctan.org/pkg/xcolor?lang=de}{xcolor Dokumentation}

\section{Fonts für mathematische Symbole}
Für spezielle Symbole in Formeln (z.~B. für Tensoren oder Matrizen) müssen zusätzliche Fonts geladen werden:
\begin{verbatim}
	\usepackage{amsmath}
	\usepackage{amsfonts} % Frakturbuchstaben
	\usepackage{euscript}[mathcal] % Schreibschrift
\end{verbatim}

Beispiel:
\begin{verbatim}
	\mathfrak{ABC}
	\mathcal{XYZ}
\end{verbatim}
$\mathfrak{ABC}$\\
$\mathcal{XYZ}$