%% LyX 1.6.7 created this file.  For more info, see http://www.lyx.org/.
%% Do not edit unless you really know what you are doing.
\documentclass[12pt,a4paper,english,ngerman,intoc,titlepage,fleqn]{scrbook}
\usepackage{lmodern}
\renewcommand{\sfdefault}{lmss}
\renewcommand{\ttdefault}{lmtt}
\usepackage[utf8]{inputenc} % linux: latin1, windows: ansinew, mac: applemac; try utf8 as well
\usepackage[T1]{fontenc}
\usepackage{fancyhdr} % fancypage Paket
\pagestyle{fancy}
\setcounter{secnumdepth}{3}
\setlength{\parskip}{\medskipamount}\setlength{\parindent}{0pt}
\usepackage[english,ngerman]{babel} % Pakete für Spracheinstellungen

\usepackage{calc} % für Berechnungen, wie Längenarithmetik in Befehlen

% mathematische Fonts
\usepackage{amsmath}
\usepackage{amsfonts} % Frakturbuchstaben
\usepackage{euscript}[mathcal] % Schreibschrift

\usepackage{xcolor} % for colorbox highlighting
\usepackage{enumitem} % mehr Möglichkeiten in enumerate der Umgebung
\usepackage{graphicx} % Paket für Beispielabbildungen
\usepackage{subcaption} % for subfigures
\usepackage[export]{adjustbox} % for subfigure alignment
\usepackage{amssymb}
\usepackage[acronym,nonumberlist]{glossaries} % Symbole und Abkuerzungen
\usepackage[unicode=true,
			bookmarks=true,
			bookmarksnumbered=true,
			bookmarksopen=true,
			bookmarksopenlevel=1,
			breaklinks=false,
			pdfborder={0 0 0},
			backref=false,
			colorlinks=false]{hyperref}

\usepackage[nameinlink]{cleveref} % Paket für Querverweise

%%%%%%%%%%%%%%%%%%%%%%%%%%%%%% User specified LaTeX commands.
% print no date
\date{}
\newcommand{\pubdate}{DATUM} % definiert das Datum der Veröffentlichung
\newcommand{\authorname}{AUTOR}

\hypersetup{pdftitle={M Sc},
			pdfauthor={\authorname},
 			pdfsubject={Dissertation zur Erlangung des Doktorgrades der Technischen Fakultät der Albert-Ludwigs-Universität Freiburg im Breisgau},
 			pdfkeywords={Ph.D. thesis},
 			pdfpagelayout=OneColumn, pdfnewwindow=true, pdfstartview=XYZ, plainpages=false}

\makeatletter

%%%%%%%%%%%%%%%%%%%%%%%%%%%%%% LyX specific LaTeX commands.
\pdfpageheight\paperheight
\pdfpagewidth\paperwidth

\DeclareRobustCommand*{\lyxarrow}{%
\@ifstar
{\leavevmode\,$\triangleleft$\,\allowbreak}
{\leavevmode\,$\triangleright$\,\allowbreak}}

% increase link area for cross-references and autoname them
\AtBeginDocument{\renewcommand{\ref}[1]{\mbox{\autoref{#1}}}}
\newlength{\abc}
\settowidth{\abc}{\space}
\addto\extrasenglish{
 \renewcommand{\equationautorefname}{\hspace{-\abc}}
 \renewcommand{\sectionautorefname}{sec.\negthinspace}
 \renewcommand{\subsectionautorefname}{sec.\negthinspace}
 \renewcommand{\subsubsectionautorefname}{sec.\negthinspace}
 \renewcommand{\figureautorefname}{Fig.\negthinspace}
 \renewcommand{\tableautorefname}{Tab.\negthinspace}} % what are these?

% that links to image floats jumps to the beginning
% of the float and not to its caption
\usepackage[figure]{hypcap}

% make caption labels bold
\setkomafont{captionlabel}{\bfseries}
\setcapindent{1em}

% fancy page Kopf-/Fußnoten Einstellungen
\renewcommand{\chaptermark}[1]{\markboth{#1}{#1}}
\renewcommand{\sectionmark}[1]{\markright{\thesection\ #1}}
\lfoot[\thepage]{} % Seitenzahlen
\rfoot[]{\thepage}
\cfoot{}

% increase the bottom float placement fraction
\renewcommand{\bottomfraction}{0.5}

% avoid that floats are placed above its sections
\let\mySection\section\renewcommand{\section}{\suppressfloats[t]\mySection}

\makeatother

% einfach: \newacronym{Schl\"ussel}{Kurzform}{Langform}
% ausf\"uhrlich: \newacronym[plural=Kurzform (plural),firstplural=Langform (plural)]{Schl\"ussel}{Kurzform}{Langform}

\newacronym{ALU}{ALU}{Albert-Ludwigs-Universität}

\newacronym{IMTEK}{IMTEK}{Institut für Mikrosystemtechnik}

\newacronym{PDF}{PDF}{Portable Document Format}
\newglossaryentry{kb}{name=\ensuremath{k_B},description=Boltzmann Konstante}
\newglossaryentry{bulkmod}{name=\ensuremath{K},description=Bulkmodul}

\newglossaryentry{density}{name=\ensuremath{\rho},description=Dichte}

\newglossaryentry{Tg}{name=\ensuremath{T_g},description=Glasübergangstemperatur}

\newglossaryentry{heaviside}{name=\ensuremath{\Theta},description=Heaviside Stufenfunktion}

\newglossaryentry{kronecker}{name=\ensuremath{\delta},description=Kroneckerdelta}

\newglossaryentry{poissonratio}{name=\ensuremath{\nu},description=Poissonzahl}
\newglossaryentry{pressure}{name=\ensuremath{p},description=Druck}

\newglossaryentry{temperature}{name=\ensuremath{T},description=Temperatur}

\newglossaryentry{youngsmod}{name=\ensuremath{E},description=Young's Modul}
\makeglossaries
\glsaddallunused

\usepackage[backend=biber,style=numeric,citestyle=numeric-comp,sorting=none]{biblatex}
\addbibresource{referenzen/Software.bib} % mehrere bib-Dateien
\addbibresource{referenzen/Potentiale.bib}      % sind möglich


\begin{document}

\subject{Dissertation zur Erlangung des Doktorgrades der Technischen Fakultät
der Albert-Ludwigs-Universität Freiburg im Breisgau}

\title{TITEL DER ARBEIT}

\author{\authorname}

\date{\pubdate}

\publishers{\includegraphics[width=0.4\columnwidth]{bilder/Uni_Logo-Grundversion_E1_A4_CMYK}\vspace{\baselineskip}\\
Albert-Ludwigs-Universität Freiburg im Breisgau\\
Technische Fakultät\\
Institut für Mikrosystemtechnik - IMTEK\vspace{-3cm}
}


\lowertitleback{\textbf{Dekan}\smallskip{}
				\\
				Prof. Dr. xxx yyy\bigskip{}
				\\
				\textbf{Referenten}\smallskip{}
				\\
				Prof. Dr. aaa bbb\smallskip{}
				\\
				Prof. Dr. xxx yyy\bigskip{}
				\\
				\textbf{Datum der Promotion (nur Notwendig für die finale Veröffentlichung)}\smallskip{}
				\\
				\pubdate}


\dedication{EINE WIDMUNG (OPTIONAL)}

\maketitle
\cleardoublepage{}

\pagestyle{plain} % hier keine Kopfzeilen etc.
% !TeX spellcheck = de_DE
\chapter*{Erklärung zur Selbstständigkeit}
Hiermit versichere ich, die vorliegende Dissertation selbständig und nur mit Hilfe der angegebenen Quellen und Hilfsmittel angefertigt zu haben.
Alle Stellen, die aus Quellen entnommen wurden, sowie alle Daten, die aus Kollaborationen stammen, sind als solche kenntlich gemacht.
Diese Arbeit hat in gleicher oder ähnlicher Form noch keiner Prüfungsbehörde vorgelegen, ein Promotionsversuch wurde von mir bisher nicht unternommen.

\bigskip{}

Freiburg, \pubdate % aus "main.tex"

\bigskip{}
\bigskip{}
\bigskip{}

\makebox[6cm]{\hrulefill}\\
(\authorname)
\thispagestyle{empty} % entfernt die Seitenzahl von dieser Seite (falls vorhanden)

\cleardoublepage{}

\pagenumbering{roman} % Beginn römischer Seitenzahlen
\pagestyle{plain} % hier keine Kopfzeilen etc.
% !TeX spellcheck = en_US
\selectlanguage{english}
\chapter*{Abstract}
\addcontentsline{toc}{chapter}{Abstract}
This document is meant to be a template for theses in the Simulation Group at \gls{IMTEK}, \gls{ALU} Freiburg.
The general layout of the document, the preamble and the loaded packages, contain all important parts of the thesis and the loaded packages contain all important functionality necessary. (Some special functions might require other~/~additional packages.)

The most important functions of each package used here are explained and the websites with full documentations linked.
\selectlanguage{ngerman}
\glsresetall
% !TeX spellcheck = de_DE
\chapter*{Kurzfassung}
\addcontentsline{toc}{chapter}{Kurzfassung}
Dieses Dokument soll als Vorlage für Abschlussarbeiten in der Arbeitsgruppe Simulation am \gls{IMTEK} der \gls{ALU} Freiburg dienen.
Der Aufbau des Dokuments, also die Präambel und die Einbindung der Kapitel etc., enthält alle wichtigen Teile der Arbeit und die hier enthaltenen Pakete können alle wichtigen Funktionen erfüllen. (Natürlich können besondere Bedürfnisse andere~/~zusätzliche Pakete erfordern.)

Die wichtigsten Funktionen der verwendeten Pakete werden hier kurz erläutert, die Websites mit den vollständigen Dokumentationen sind mit verlinkt.
\glsresetall
\pagestyle{fancy} % ab hier wieder Kopfzeilen etc.

\cleardoublepage{}

\lhead[\fancyplain{}{Veröffentlichungen und Präsentationen}]{\fancyplain{}{}}
\rhead[\fancyplain{}{}]{\fancyplain{}{Veröffentlichungen und Präsentationen}}
% !TeX spellcheck = de_DE
\chapter*{Veröffentlichungen und Vorträge}
\addcontentsline{toc}{chapter}{Veröffentlichungen und Präsentationen}
\label{chap:veroeffentlichungen}
Eine Liste der Veröffentlichungen und Vorträge, die im Rahmen der Promotion entstanden sind. Der erste Eintrag dient als Beispiel.

\section*{Journalveröffentlichungen in dieser Arbeit}
\begin{itemize}
  \item \textbf{Richard Jana}, Lars Pastewka, \textit{Correlations of non-affine displacements in metallic glasses through the yield transition}, Journal of Physics: Materials, 2(4), 045006 (2019). --- \textbackslash cref\{chap:CuZr\}
  \item Autoren (eigener Name \textbf{fett} gedruckt), \textit{Titel der Veröffentlichung}, Journal, Volume(Number), Seiten (Jahr). --- \textbackslash cref\{dazugehöriges Kapitel\}
  \item noch eine Veröffentlichung --- \cref{Kap:Struktur}
  \item[] Bei manchen Veröffentlichungen sollte dargestellt werden, welchen Anteil die Autoren hatten.
  \item und noch eine Veröffentlichung --- \cref{Kap:Pakete}
\end{itemize}


\section*{Andere Journalveröffentlichungen}
\begin{itemize}
  \item Veröffentlichung, die nicht in diese Arbeit eingeflossen ist (z.~B. weil der eigene Anteil geringer war).
\end{itemize}

\section*{Konferenzvorträge}
\begin{itemize}
  \item \textbf{\authorname}, weitere Autoren, \textit{Titel des Vortrags}, Konferenz, Ort, Land (Datum)
\end{itemize}


\cleardoublepage{}

\rhead[\fancyplain{}{}]{\fancyplain{}{\rightmark}} % Nummer und Titel des Abschnitts
\lhead[\fancyplain{}{\leftmark}]{\fancyplain{}{}} % Titel des Kapitels
\tableofcontents{}
\listoffigures 
\listoftables

\cleardoublepage{}

\pagenumbering{arabic} % Beginn arabischer Seitenzahlen
\rhead[\fancyplain{}{\leftmark}]{\fancyplain{}{\rightmark}} % Titel des Kapitels, Nummer und Titel des Abschnitts
\lhead[\fancyplain{}{\chaptername\;\thechapter}]{\fancyplain{}{}} % "Kapitel" + Nummer des Kapitels
% !TeX spellcheck = de_DE
\chapter{Dokumentstruktur}\label{Kap:Struktur}

\section{Kopf- und Fußzeilen}
Kopfzeilen (und Fußzeilen) können mit dem Paket fancyhdr gesteuert werden. Zwischen Seiten im einfachen Layout und im fancyhdr Layout wird mit den Befehlen
\begin{verbatim}
	\pagestyle{plain}
	\pagestyle{fancy}
\end{verbatim}
gewechselt.

In \glqq{}main.tex\grqq{} definiert, gilt die jeweilige Definition solange bis eine andere folgt. Die Syntax ist wie folgt:
\begin{verbatim}
	\lhead[\fancyplain{speziell}{normal}]{\fancyplain{}{}} % rechter Teil
	\rhead[\fancyplain{}{}]{\fancyplain{}{}} % linker Teil
\end{verbatim}
Es wird unterschieden zwischen normalen und speziellen Seiten (z.~B. die erste Seite eines Kapitels), geraden (in [ ]) und ungeraden (in\{ \}) Seiten und zwischen linkem (\textbackslash lhead) und rechtem Teil (\textbackslash rhead) einer Kopfzeile.

Fußzeilen werden analog eingefügt mit
\begin{verbatim}
	\lfoot[\fancyplain{}{}]{\fancyplain{}{}}
	\rfoot[\fancyplain{}{}]{\fancyplain{}{}}
\end{verbatim}

Neben manuell eingefügtem Text existieren einige Möglichkeiten für variablen Text in den Kopfzeilen:
\begin{verbatim}
	\thechapter % Nummer des aktuellen Kapitels
	\thesection % Nummer des aktuellen Abschnitts
	\chaptername % "Kapitel" (in der aktuellen Sprache)
	\leftmark % Name und Nummer des aktuellen Kapitels
	\rightmark % Name und Nummer des aktuellen Abschnitts
\end{verbatim}

\hyperlink{https://ctan.org/pkg/fancyhdr?lang=de}{fancyhdr Dokumentation}

\section{Spracheinstellungen}
Die Spracheinstellungen bestehen hier aus zwei Teilen: Zum einen der Kodierung der Dateien, die es ermöglicht Umlaute direkt einzugeben (\glqq{}ä\grqq{} anstatt \glqq{}\textbackslash "a\grqq{}). Dazu werden die beiden Befehle
\begin{verbatim}
	\usepackage[Kodierung]{inputenc}
	\usepackage[T1]{fontenc}
\end{verbatim}
benötigt. Statt Kodierung muss die entsprechende Kodierung eingetragen werden: Entweder \glqq{}utf8\grqq{}, oder je nach Betriebssystem \glqq{}latin1\grqq{} (Linux), \glqq{}ansinew\grqq{} (Windows) oder \glqq{}applemac\grqq{} (Mac).

Die Trennung von Wörtern, die Formatierung des Datums und Begriffe wie \glqq{}Inhaltsverzeichnis\grqq{} (anstatt englisch \glqq{}Contents\grqq{}) werden als Paket geladen:
\begin{verbatim}
	\usepackage[english,ngerman]{babel}
\end{verbatim}

Englisch wird hier mit geladen, da es z.~B. im Abstract benötigt wird. Dort wird es mit
\begin{verbatim}
	\selectlanguage{english}
	\selectlanguage{ngerman}
\end{verbatim}
geladen und anschließend wieder durch Deutsch ersetzt.
\glsresetall
% !TeX spellcheck = de_DE
\chapter{Pakete}\label{Kap:Pakete}
Dieses Kapitel beschreibt einige der verwendeten Pakete. Vollständige Dokumentationen sind verlinkt.

\section{glossaries}\label{Abs:glossaries}
Dieses Paket ermöglicht es gleichzeitig eine Liste von Abkürzungen und eine Liste von Symbolen zu führen. In \glqq{}main.tex\grqq{} wird das Paket geladen:
\begin{verbatim}
	\usepackage[acronym,nonumberlist]{glossaries}
	\makeglossaries
\end{verbatim}

die Dateien mit den Listen eingebunden
\begin{verbatim}
	% einfach: \newacronym{Schl\"ussel}{Kurzform}{Langform}
% ausf\"uhrlich: \newacronym[plural=Kurzform (plural),firstplural=Langform (plural)]{Schl\"ussel}{Kurzform}{Langform}

\newacronym{ALU}{ALU}{Albert-Ludwigs-Universität}

\newacronym{IMTEK}{IMTEK}{Institut für Mikrosystemtechnik}

\newacronym{PDF}{PDF}{Portable Document Format}
	\newglossaryentry{kb}{name=\ensuremath{k_B},description=Boltzmann Konstante}
\newglossaryentry{bulkmod}{name=\ensuremath{K},description=Bulkmodul}

\newglossaryentry{density}{name=\ensuremath{\rho},description=Dichte}

\newglossaryentry{Tg}{name=\ensuremath{T_g},description=Glasübergangstemperatur}

\newglossaryentry{heaviside}{name=\ensuremath{\Theta},description=Heaviside Stufenfunktion}

\newglossaryentry{kronecker}{name=\ensuremath{\delta},description=Kroneckerdelta}

\newglossaryentry{poissonratio}{name=\ensuremath{\nu},description=Poissonzahl}
\newglossaryentry{pressure}{name=\ensuremath{p},description=Druck}

\newglossaryentry{temperature}{name=\ensuremath{T},description=Temperatur}

\newglossaryentry{youngsmod}{name=\ensuremath{E},description=Young's Modul}
\end{verbatim}

Um alle Abkürzungen in der Liste der Abkürzungen zu drucken, auch solche die nicht verwendet wurden, dient dieser Befehl:
\begin{verbatim}
	\glsaddallunused
\end{verbatim}
Bei der Erstellung des Dokuments kann dies hilfreich sein, um alle bereits definierten Abkürzungen im \gls{PDF} zu sehen.

Diese Befehlen fügen die Listen in das Dokument ein:
\begin{verbatim}
	\printglossary[type=\acronymtype,title=Abkürzungen]
	\printglossary[title=Symbole]
\end{verbatim}

Die Liste der Symbole wird in \glqq{}Symbole.tex\grqq{} definiert. Die Syntax ist
\begin{verbatim}
	\newglossaryentry{Schlüssel}{name=\ensuremath{Symbol},description=Beschreibungstext}
\end{verbatim}
(Der Befehl ensuremath erleichtert später die Verwendung des Symbols im Text und in Gleichungen. Würde hier eine \$-Umgebung verwendet, würden in Gleichungen Fehler auftreten.)

Abkürzungen werden in \glqq{}Abkuerzungen.tex\grqq{} definiert mit
\begin{verbatim}
	\newacronym{Schlüssel}{Kurzform}{Langform}
	\newacronym[plural=Kurzform (plural),firstplural=Langform (plural)]{Schlüssel}{Kurzform}{Langform}
\end{verbatim}
Die erste Variante definiert den Plural nicht ausdrücklich, d.~h. er wird automatisch erzeugt. Entspricht dies nicht der gewünschten Form, kann diese manuell definiert werden.

Abkürzungen und Symbole werden im Text mit dem Befehl
\begin{verbatim}
	\gls{Schlüssel}
\end{verbatim}
eingefügt. Dabei wird in den \{\} der Schlüssel für die jeweilige Abkürzung verwendet. Der Plural der Abkürzung wird mit
\begin{verbatim}
	\glspl{Schlüssel}
\end{verbatim}
eingefügt. Beispiele für ein Symbol im Text und in einer Gleichung, Singular und Plural Abkürzungen (\glqq{}PDF\grqq{} wurde bereits verwendet und erscheint deshalb hier in der Kurzform):
\begin{verbatim}
	\gls{density}
	\begin{equation}
		\gls{density} = \frac{m}{V}
		\label{Gl:dichte}
	\end{equation}
	\gls{IMTEK}
	\glspl{PDF}
\end{verbatim}
\gls{density}
\begin{equation}
	\gls{density} = \frac{m}{V}
	\label{Gl:dichte}
\end{equation}
\gls{IMTEK}\\
\glspl{PDF}

Die Langform einer Abkürzung wird nur bei der ersten Verwendung der Abkürzung gedruckt, danach die Kurzform. Um später erneut die Langform zu drucken (z.~B. in jedem Kapitel), kann der Zähler für einzelne oder alle Abkürzung zurück gesetzt werden:
\begin{verbatim}
	\glsreset{Schlüssel}
	\glsresetall
\end{verbatim}

Vor dem Erzeugen des \glspl{PDF} muss in einer Kommandozeile, im Verzeichnis mit \glqq{}main.tex\glqq{}, folgender Befehl ausgeführt werden (sonst fehlen die Listen im \gls{PDF}):
\begin{verbatim}
	makeglossaries main
\end{verbatim}
(\glqq{}main\grqq{} ist hier der Name der Hauptdatei, ohne die Endung \glqq{}tex\grqq{})

\hyperlink{https://ctan.org/pkg/glossaries?lang=de}{glossaries Dokumentation}

\section{cleveref}\label{Abs:cleveref}
Objekte innerhalb der Arbeit (Abbildungen, Kapitel, Gleichungen) können referenziert werden, wenn ihnen zuvor ein Schlüssel zugewiesen wurde. Dies erfolgt direkt am Objekt:
\begin{verbatim}
	\section{cleveref}\label{Abs:cleveref}
\end{verbatim}

Im Text kann dann auf dieses Objekt verwiesen werden:
\begin{verbatim}
	\cref{Abs:cleveref}
	\cref{Tab:beispiel}
	\cref{Abb:beispiel}
\end{verbatim}
\cref{Abs:cleveref}\\
\cref{Tab:beispiel}\\
\cref{Abb:beispiel}

Auch ein ganzer Bereich von Objekten kann referenziert werden:
\begin{verbatim}
	\crefrange{Abs:glossaries}{Abs:cleveref}
\end{verbatim}
\crefrange{Abs:glossaries}{Abs:cleveref}

(Es ist gute Praxis die Schlüssel so zu vergeben, dass daraus ersichtlich ist welcher Art das Objekt ist (Abbildung, Kapitel, etc.), da sich dadurch die Lesbarkeit des Quelltextes erhöht.)

\hyperlink{https://ctan.org/pkg/cleveref?lang=de}{cleveref Dokumentation}

\section{xcolor}
Dieses Paket erlaubt es die \colorbox{blue}{F}\colorbox{green}{a}\colorbox{yellow}{r}\colorbox{orange}{b}\colorbox{red}{e} von Schrift, Hintergrund, Rahmen oder ganzen Seiten zu ändern. Kommentare im Text können z.~B. farbig hinterlegt werden, um die Sichtbarkeit zu erhöhen:
\begin{verbatim}
	\colorbox{yellow}{hier Zitat einfügen}
\end{verbatim}
\colorbox{yellow}{hier Zitat einfügen}

\hyperlink{https://ctan.org/pkg/xcolor?lang=de}{xcolor Dokumentation}

\section{Fonts für mathematische Symbole}
Für spezielle Symbole in Formeln (z.~B. für Tensoren oder Matrizen) müssen zusätzliche Fonts geladen werden:
\begin{verbatim}
	\usepackage{amsmath}
	\usepackage{amsfonts} % Frakturbuchstaben
	\usepackage{euscript}[mathcal] % Schreibschrift
\end{verbatim}

Beispiel:
\begin{verbatim}
	\mathfrak{ABC}
	\mathcal{XYZ}
\end{verbatim}
$\mathfrak{ABC}$\\
$\mathcal{XYZ}$
\glsresetall
% !TeX spellcheck = de_DE
\chapter{Zitate}

\section{Syntax}
Die Informationen über die zitierten Quellen müssen als \textasteriskcentered .bib-Dateien bereitgestellt und in \glqq{} main.tex\grqq{} geladen werden:
\begin{verbatim}
	\addbibresource{referenzen/Software.bib}
	\addbibresource{referenzen/Potentiale.bib}
\end{verbatim}
Dabei dürfen alle Quellen in einer Datei gespeichert sein, aber auch auf mehrere Dateien verteil werden.

Quellen können entweder einzeln zitiert werden, oder als Gruppe:
\begin{verbatim}
	\cite{Schlüssel}
	\cite{Schlüssel_1,Schlüssel_2,Schlüssel_3,Schlüssel_4}
\end{verbatim}

Um Zitate einzufügen wie im Beispiel unten sollte ein geschütztes Leerzeichen (\textasciitilde) verwendet werden, um Zeilenumbrüche am Zitat zu verhindern:
\begin{verbatim}
	LAMMPS~\cite{Plimpton1995}
\end{verbatim}

In das Dokument eingefügt wird die Bibliographie in \glqq{} main.tex\grqq{} mit
\begin{verbatim}
	\printbibliography
\end{verbatim}

\section{Anpassungsmöglichkeiten}
In \glqq{} main.tex\grqq{} gibt es verschiedene Möglichkeiten die Art der Zitierung zu beeinflussen.
\begin{verbatim}
	\usepackage[backend=biber,style=numeric,citestyle=numeric-comp,sorting=none]{biblatex}
\end{verbatim}
Hier wurde festgelegt die Zitate mit Zahlen zuzuordnen (\glqq{}style=numeric\grqq{}), mehrere Zitate an einer Stelle kompakt zu beschreiben (\glqq{}citestyle=numeric-comp\grqq{}) und die Quellen in der Reihenfolge aufzulisten in der sie im Text auftauchen anstatt sie zu sortieren (\glqq{}sorting=none\grqq{}).

\section{Beispiel - Softwares und Potentiale}
Als Beispiel sind hier einige Softwares genannt, die in unserer Gruppe häufig verwendet werde: LAMMPS~\cite{Plimpton1995}, OVITO~\cite{Stukowski2010}, ASE~\cite{HjorthLarsen2017}, matscipy~\cite{matscipy} und voro++~\cite{Rycroft2009}.
Häufig verwendete interatomare Potentiale für amorphen Kohlenstoff~\cite{Deringer2016,Pastewka:2013p205410,Liyanage2014} und CuZr metallisches Glas~\cite{Mendelev2009,Cheng2009}.
\glsresetall
% !TeX spellcheck = de_DE
\chapter{Abbildungen, Tabellen \& Vermischtes}

\section{Abbildungen}
Beispiel für eine Abbildung:
\begin{verbatim}
	\begin{figure}
  	  \centering
  	  \begin{subfigure}[h!]{0.15\linewidth}
    	\includegraphics[width=\linewidth]{example-image-a}
    	\caption{}
  	  \end{subfigure}
  	  \begin{subfigure}[h!]{0.15\linewidth}
    	\includegraphics[width=\linewidth]{example-image-b}
    	\caption{}
  	  \end{subfigure}\\
  	  \begin{subfigure}[h!]{0.3\linewidth}
    	\includegraphics[width=\linewidth]{example-image-c}
    	\caption{}
  	  \end{subfigure}
  	  \caption[Kurzbeschreibung für Abbildungsverzeichnis]{Lange Beschreibung,
  	  			die unter dem Bild angezeigt wird.}
  	  \label{Abb:beispiel}
	\end{figure}
\end{verbatim}

\begin{figure}
  \centering
  \begin{subfigure}[h!]{0.25\linewidth}
    \includegraphics[width=\linewidth]{example-image-a}
    \caption{}
  \end{subfigure}
  \begin{subfigure}[h!]{0.25\linewidth}
    \includegraphics[width=\linewidth]{example-image-b}
    \caption{}
  \end{subfigure}\\
  \begin{subfigure}[h!]{0.5\linewidth}
    \includegraphics[width=\linewidth]{example-image-c}
    \caption{}
  \end{subfigure}
  \caption[Kurzbeschreibung für Abbildungsverzeichnis]{Lange Beschreibung, die unter dem Bild angezeigt wird.}
  \label{Abb:beispiel}
\end{figure}

Zeilenumbrüchen funktionieren auch hier mit \textbackslash \textbackslash . Der leere Befehl \textbackslash caption\{\} wird benötigt um die Nummerierung \glqq{}(a)\grqq{} usw. einzufügen. Anstatt von \glqq{}example-image-a\grqq{} usw. muss der Pfad zum gewünschten Bild eingefügt werden.

\pagebreak % damit Abbildung und Tabelle hier nicht durcheinander geraten

\section{Tabellen}
Beispiel für eine Tabelle:
\begin{verbatim}
	\begin{table}
	\centering
	\caption[Kurzbeschreibung für Tabellenverzeichnis]{Lange Beschreibung,
				die über der Tabelle angezeigt wird.}
	\begin{tabular}{ccc}
	Spalte 1 & Spalte 2 & Spalte 3\\
	\hline 
	A & B & C \\
	D & E & F \\
	G & H & I \\
	\hline 
	\end{tabular}
	\label{Tab:beispiel}
	\end{table}
\end{verbatim}

\begin{table}
\centering
\caption[Kurzbeschreibung für Tabellenverzeichnis]{Lange Beschreibung, die über der Tabelle angezeigt wird.}
\begin{tabular}{ccc}
Spalte 1 & Spalte 2 & Spalte 3\\
\hline 
A & B & C \\
D & E & F \\
G & H & I \\
\hline 
\end{tabular}
\label{Tab:beispiel}
\end{table}

\section{Größen und Einheiten im Text}
Größen mit Einheiten sollten im Text so verwendet werden:
\begin{verbatim}
	\gls{kb} $= 1,38064852 10^{-23}$~m$^2$kgs$^{-2}$K$^{-1}$
	$l = 1.85$~\AA{}
\end{verbatim}
\gls{kb} $= 1,38064852 10^{-23}$~m$^2$kgs$^{-2}$K$^{-1}$\\
$l = 1.85$~\AA{}
\glsresetall

\lhead[\fancyplain{}{Zusammenfassung}]{\fancyplain{}{}}
\rhead[\fancyplain{}{}]{\fancyplain{}{Zusammenfassung}}
\chapter*{Zusammenfassung}
\addcontentsline{toc}{chapter}{Zusammenfassung}

Zusammenfassung des Inhalts.


\glsresetall

\cleardoublepage{}

\appendix % Beginn des Anhangs
\lhead[\fancyplain{}{Anhang}]{\fancyplain{}{}}
\rhead[\fancyplain{}{}]{\fancyplain{}{Anhang}}
\chapter{Anhang}
\section{Erster Teil}
\label{Abs:anhang_1}

Zusätzliche Erklärungen zu Berechnungen und Formeln, Abbildungen, ...

\cleardoublepage{}

\lhead[\fancyplain{}{Abkürzungen \& Symbole}]{\fancyplain{}{}}
\rhead[\fancyplain{}{}]{\fancyplain{}{Abkürzungen \& Symbole}}
\printglossary[type=\acronymtype,title=Abkürzungen]
\addcontentsline{toc}{chapter}{Abkürzungen \& Symbole}
\printglossary[title=Symbole]

\cleardoublepage{}

\lhead[\fancyplain{}{Danksagung}]{\fancyplain{}{}}
\rhead[\fancyplain{}{}]{\fancyplain{}{Danksagung}}
% !TeX spellcheck = de_DE
\chapter*{Danksagung}
\addcontentsline{toc}{chapter}{Danksagung}
Menschen~/~Institutionen, denen man danken könnte (in etwa der richtigen Reihenfolge)

\begin{itemize}
  \item Lars
  \item Zweitgutachter Professor
  \item Koautoren der Veröffentlichungen
  \item Arbeitsgruppe (im Allgemeinen, oder auch einzelnen Personen)
  \item Eltern
  \item Rechenzeit~/~Finanzierung: Jülich, NEMO, BW-HPC, DFG, ... (falls zutreffend)
\end{itemize}









\cleardoublepage{}

\lhead[\fancyplain{}{\leftmark}]{\fancyplain{}{}}
\rhead[\fancyplain{}{}]{\fancyplain{}{\leftmark}}
\printbibliography
\addcontentsline{toc}{chapter}{Literatur}

\end{document}