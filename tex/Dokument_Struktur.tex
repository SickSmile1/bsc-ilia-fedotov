% !TeX spellcheck = de_DE
\chapter{Dokumentstruktur}\label{Kap:Struktur}

\section{Kopf- und Fußzeilen}
Kopfzeilen (und Fußzeilen) können mit dem Paket fancyhdr gesteuert werden. Zwischen Seiten im einfachen Layout und im fancyhdr Layout wird mit den Befehlen
\begin{verbatim}
	\pagestyle{plain}
	\pagestyle{fancy}
\end{verbatim}
gewechselt.

In \glqq{}main.tex\grqq{} definiert, gilt die jeweilige Definition solange bis eine andere folgt. Die Syntax ist wie folgt:
\begin{verbatim}
	\lhead[\fancyplain{speziell}{normal}]{\fancyplain{}{}} % rechter Teil
	\rhead[\fancyplain{}{}]{\fancyplain{}{}} % linker Teil
\end{verbatim}
Es wird unterschieden zwischen normalen und speziellen Seiten (z.~B. die erste Seite eines Kapitels), geraden (in [ ]) und ungeraden (in\{ \}) Seiten und zwischen linkem (\textbackslash lhead) und rechtem Teil (\textbackslash rhead) einer Kopfzeile.

Fußzeilen werden analog eingefügt mit
\begin{verbatim}
	\lfoot[\fancyplain{}{}]{\fancyplain{}{}}
	\rfoot[\fancyplain{}{}]{\fancyplain{}{}}
\end{verbatim}

Neben manuell eingefügtem Text existieren einige Möglichkeiten für variablen Text in den Kopfzeilen:
\begin{verbatim}
	\thechapter % Nummer des aktuellen Kapitels
	\thesection % Nummer des aktuellen Abschnitts
	\chaptername % "Kapitel" (in der aktuellen Sprache)
	\leftmark % Name und Nummer des aktuellen Kapitels
	\rightmark % Name und Nummer des aktuellen Abschnitts
\end{verbatim}

\hyperlink{https://ctan.org/pkg/fancyhdr?lang=de}{fancyhdr Dokumentation}

\section{Spracheinstellungen}
Die Spracheinstellungen bestehen hier aus zwei Teilen: Zum einen der Kodierung der Dateien, die es ermöglicht Umlaute direkt einzugeben (\glqq{}ä\grqq{} anstatt \glqq{}\textbackslash "a\grqq{}). Dazu werden die beiden Befehle
\begin{verbatim}
	\usepackage[Kodierung]{inputenc}
	\usepackage[T1]{fontenc}
\end{verbatim}
benötigt. Statt Kodierung muss die entsprechende Kodierung eingetragen werden: Entweder \glqq{}utf8\grqq{}, oder je nach Betriebssystem \glqq{}latin1\grqq{} (Linux), \glqq{}ansinew\grqq{} (Windows) oder \glqq{}applemac\grqq{} (Mac).

Die Trennung von Wörtern, die Formatierung des Datums und Begriffe wie \glqq{}Inhaltsverzeichnis\grqq{} (anstatt englisch \glqq{}Contents\grqq{}) werden als Paket geladen:
\begin{verbatim}
	\usepackage[english,ngerman]{babel}
\end{verbatim}

Englisch wird hier mit geladen, da es z.~B. im Abstract benötigt wird. Dort wird es mit
\begin{verbatim}
	\selectlanguage{english}
	\selectlanguage{ngerman}
\end{verbatim}
geladen und anschließend wieder durch Deutsch ersetzt.